\documentclass[12pt, a4paper, oneside, fontset=windows]{ctexart}
\usepackage{amsmath, amsthm, amssymb, appendix, bm, graphicx, hyperref, mathrsfs}

\title{\textbf{最优化第一次作业}}
\author{付星桐}
\date{\today}
\linespread{1.5}

\newtheorem{theorem}{定理}[section]
\newtheorem{definition}[theorem]{定义}
\newtheorem{lemma}[theorem]{引理}
\newtheorem{corollary}[theorem]{推论}
\newtheorem{example}[theorem]{例}
\newtheorem{proposition}[theorem]{命题}

\renewcommand{\abstractname}{\Large\textbf{摘要}}

\begin{document}
\pagestyle{empty}
\maketitle

\setcounter{page}{0}
\maketitle
\thispagestyle{empty}


\begin{abstract}
    忆梅下西洲,折梅寄江北。 
    \par\textbf{关键词:}爱情; 诗歌. 
\end{abstract}


\newpage
% 添加目录
\pagenumbering{Roman}
\setcounter{page}{1}
\tableofcontents
\newpage
\setcounter{page}{1}
\pagenumbering{arabic}

\newpage
\section{作业}

\subsection{保凸运算1}
$aC+b=\left \{  x|x=ay+b,y\in C  \right \} $

$\forall x{_{1}}^{},x{_{2}}^{}\in aC+b,\exists y{_{1}}^{},y{_{2}}^{}\in C $

使$ \begin{cases}
    & \text{  } x{_{1}}^{}=ay{_{1}}^{}+b \\ 
    & \text{  } x{_{2}}^{}= ay{_{2}}^{}+b
    \end{cases}$

$ \forall \theta \in \left [  0,1\right ]$

$\theta x{_{1}}^{}+\left (1- \theta \right )x{_{2}}^{}=\theta \left (  ay{_{1}}^{}+b \\\right ) +\left (1- \theta \right ) \left ( ay{_{2}}^{}+b \\ \right )
=a\left [ \theta y{_{1}}^{}+\left (1- \theta \right )y{_{2}}^{}\right ]+b $

又$\theta y{_{1}}^{}+\left (1- \theta \right )y{_{2}}^{}\in C$

所以原式=$a\left [ \theta y{_{1}}^{}+\left (1- \theta \right )y{_{2}}^{}\right ]+b\in aC+b$

即aC+b也是凸集

\subsection{保凸运算2}
$g(x)=f(Ax+b) ,dom\; g=\left \{  x|Ax+b\in  dom\; f  \right \}$

证明:

令$x,y\in dom\; g,0\leq \theta \leq 1;$

$g(\theta x+\left (1- \theta \right )y) =f(\theta Ax+\left (1- \theta \right )Ay+b)=f(\theta \left ( Ax+b \right )+\left (1- \theta \right )\left ( Ay+b \right ))\leq \theta f(Ax+b)+\left (1- \theta \right )f(Ay+b)=\theta g(x)+\left (1- \theta \right ) g(y)$
  
根据凸函数定义,g(x)是凸函数,仿射映射得证
   
\subsection{保凸运算3}
$f(x)=g(Ax+b) ,dom\; f=\left \{  x|Ax+b\in  dom\; g  \right \}$

证明:

令$x,y\in dom\; f,0\leq \theta \leq 1;$

已知$g(\theta x+\left (1- \theta \right )y) \leq \theta g(x)+\left (1- \theta \right ) g(y)$

又$g(\theta x+\left (1- \theta \right )y)=f(\theta \left ( Ax+b \right )+\left (1- \theta \right )\left ( Ay+b \right )),
\theta g(x)+\left (1- \theta \right ) g(y)=\theta f(Ax+b)+\left (1- \theta \right )f(Ay+b)$

所以易得:$f(\theta \left ( Ax+b \right )+\left (1- \theta \right )\left ( Ay+b \right ))\leq \theta f(Ax+b)+\left (1- \theta \right )f(Ay+b)$

根据凸函数定义,f(x)是凸函数,得证
   
\subsection{证明无穷范数满足范数的三个性质}
证明:

(1)正定性

$\Vert x \Vert_\infty =\underset{i}{max}\left | x_{i} \right |\geq 0$

$if \;\Vert x \Vert_\infty = \underset{i}{max}\left | x_{i} \right |=0,$

$0\leq \left | x_{i} \right |\leq \underset{i}{max}\left | x_{i} \right |=0$

$so\;  \left | x_{i} \right |=0$

$so:if \;\Vert x \Vert_\infty = 0,only\; while \; x_{i}=0$

(2)奇次性

$\Vert tx \Vert_\infty=\underset{i}{max}\left | tx_{i} \right |=\left | t \right |\underset{i}{max}\left | x_{i} \right |
=\left | t \right |\Vert x \Vert_\infty$

(3)三角不等式

$\Vert x+y \Vert_\infty=\underset{i}{max}\left | x_{i}+y_{i} \right |,
\Vert x \Vert_\infty=\underset{i}{max}\left | x_{i} \right |,
\Vert y \Vert_\infty=\underset{i}{max}\left | y_{i} \right |,$

$\Vert x+y \Vert_\infty^{2}-\left (  \Vert x \Vert_\infty+\Vert y \Vert_\infty\right )^{2}=
\left ( \underset{i}{max}\left | x_{i}+y_{i} \right | \right )^{2}-\left (  \underset{i}{max}\left | x_{i} \right |+\underset{i}{max}\left | y_{i} \right |\right )^{2}=\underset{i}{max}\; \left ( x_{i}+y_{i} \right )^{2}-\underset{i}{max}\; x_{i}^{2}-\underset{i}{max}\; y_{i}^{2}-2\underset{i}{max}\left | y_{i} \right |\underset{i}{max}\left | x_{i} \right |=2\underset{i}{max}\;x_{i}y_{i}-2\underset{i}{max}\left | y_{i} \right |\underset{i}{max}\left | x_{i} \right |\leq 0$

所以$\Vert x+y \Vert_\infty\leq \left (  \Vert x \Vert_\infty+\Vert y \Vert_\infty\right )$

\subsection{证明向量范数均是凸函数}

$f(x)=\Vert x \Vert$

$\forall \theta \in \left [  0,1\right ],\forall x,y\in domf$

$f(\theta x+\left (1- \theta \right )y)=\Vert \theta x+\left (1- \theta \right )y \Vert\leq \Vert \theta x\Vert+\Vert \left (1- \theta \right )y\Vert=\theta \Vert x\Vert+\left (1- \theta \right )\Vert y \Vert=\theta f(x)+\left (1- \theta \right )f(y)$

满足凸函数的定义,向量范数均是凸函数

\subsection{证明Log-sum-exp函数是凸函数}
$f(x)=log\sum_{k=1}^{n}e^{x_{k}}$

又$\underset{i}{max}\; x_{i}\leq log\sum_{k=1}^{n}e^{x_{k}}\leq \underset{i}{max}\; x_{i}+logn$,求得:

$\nabla^{2} f(x)=\frac{1}{\left ( \mathbf{1}^\mathrm{T}\mathbf{z} \right )^{2}}\left (  \left ( \mathbf{1}^\mathrm{T}\mathbf{z} \right )diag\left ( \mathbf{z} \right )-\mathbf{z}\mathbf{z}^\mathrm{T}\right )
\; where\; \mathbf{z}=\left (  \begin{matrix}
e^{x_{1}} ,& e^{x_{2}}  ,& \cdot \cdot \cdot \cdot \cdot \cdot e^{x_{k-1}} , & e^{x_{k}} 
\end{matrix}\right )$

根据柯西不等式:$\left ( \mathbf{a}^\mathrm{T}\mathbf{a} \right )\left ( \mathbf{b}^\mathrm{T}\mathbf{b} \right )\geq \left ( \mathbf{a}^\mathrm{T}\mathbf{b} \right )^{2}$

for all v,

$\mathbf{v}^\mathrm{T}\nabla^{2}f(x)\mathbf{v}=\frac{1}{\left ( \mathbf{1}^\mathrm{T}\mathbf{z} \right )^{2}}\left (  \left (\sum_{i=1}^{n}z_{i}\right )\left ( (\sum_{i=1}^{n}v_{i}^{2}z_{i} \right )-\left (  (\sum_{i=1}^{n}v_{i}z_{i}\right )^{2}\right )\geq 0
\; with\; a_{i}=v_{i}\sqrt{z_{i}},b_{i}=\sqrt{z_{i}}$

满足凸函数的二阶条件,该函数是凸函数

\subsection{严格凸函数解集唯一}

用反证法:

若严格凸函数不只有一个最小值点,即有两个最小值点,设为x,y

则由严格凸函数定义有:$f(\theta x+\left (1- \theta \right )y)\leq \theta f(x)+ \left (1- \theta \right )f(y)-\frac{m}{2} \theta\left (1- \theta \right )\Vert x-y \Vert^{2}=f(x)-\frac{m}{2} \theta\left (1- \theta \right )\Vert x-y \Vert^{2}< f(x)$

与x是最小值点矛盾,所以严格凸函数解集唯一

\subsection{一阶必要性定理证明}
$if: \nabla f(x^{*})\neq 0,d=-\nabla f(x^{*})$

$then: \nabla f(\mathbf{x} ^{*})^\mathrm{T}d=-\nabla f(\mathbf{x} ^{*})^\mathrm{T}\nabla f(\mathbf{x} ^{*})=-\Vert \nabla f(\mathbf{x} ^{*}) \Vert^{2}< 0$

$\exists \sigma > 0,\: while\; \lambda \in \left ( 0,\sigma  \right ),f(\mathbf{x} ^{*}+\lambda d)< f(\mathbf{x} ^{*})$

与x*处取得局部最小解矛盾,所以必要性成立,局部最小点处,一定有$\nabla f(\mathbf{x} ^{*})=  0$
% 参考文献
\newpage

\begin{thebibliography}{99}
    \bibitem{a}作者. \emph{文献}[M]. 地点:出版社,年份.
    \bibitem{b}作者. \emph{文献}[M]. 地点:出版社,年份.
\end{thebibliography}


% 附录
\newpage

\begin{appendices}
    \renewcommand{\thesection}{\Alph{section}}
    \section{附录标题}
        这里是附录. 
\end{appendices}

\end{document}