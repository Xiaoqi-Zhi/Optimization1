\documentclass[12pt, a4paper, oneside, fontset=windows]{ctexart}
\usepackage{amsmath, amsthm, amssymb, appendix, bm, graphicx, hyperref, mathrsfs}
\usepackage[normalem]{ulem}
\useunder{\uline}{\ul}{}
\title{\textbf{最优化第二次作业}}
\author{大数据001\\郅啸淇\\学号:2184114639}
\date{\today}
\linespread{1.5}

\newtheorem{theorem}{定理}[section]
\newtheorem{definition}[theorem]{定义}
\newtheorem{lemma}[theorem]{引理}
\newtheorem{corollary}[theorem]{推论}
\newtheorem{example}[theorem]{例}
\newtheorem{proposition}[theorem]{命题}
\begin{document}
\maketitle
\newpage
\tableofcontents
\newpage
\section{强凸函数相互等价性}
\subsection{Monotonicity}
因为$\triangledown g(x)$为单调函数,故有$(\triangledown g(y) - \triangledown g(x))^{T}(y - x) \geq  0$

代入$g(x) = f(x) - \frac{m}{2}x^{T}x$,有$(\triangledown f(y) - my -\triangledown f(x) + mx)^{T}(y-x) \geq 0$

即$((\triangledown f(y) - \triangledown f(x)) - m(y -x))^{T}(y-x) \geq 0$

即$(\triangledown f(y) - \triangledown f(x))^{T} \geq m\| y - x\|^{2}_{2} $

Monotonicity得证
\subsection{Jensen's Inequality}
由$g(x)$为凸函数,有$g(\theta x + (1-\theta )y) \leq \theta g(x) + (1-\theta)g(y)$,代入$g(x) = f(x) - \frac{m}{2}x^{T}x$有

$f(\theta x + (1-\theta)y) \leq \theta f(x) + (1-\theta) f(y) - \frac{m}{2}(\theta x^{T}x + (1-\theta) y^{T}y) + \frac{m}{2}(\theta x + (1-\theta)y)^{T}(\theta x + (1-\theta)y)$

即为$f(\theta x + (1-\theta)y) \leq \theta f(x) + (1-\theta) f(y) - \frac{m}{2}(1-\theta)\theta(x^{T}x + y^{T}y -(xy^{T}+x^{T}y))$

即为$f(\theta x + (1-\theta)y) \leq \theta f(x) + (1-\theta) f(y) - \frac{m}{2}(1-\theta)\theta \|x - y\|^{2}_{2}$

Jensen's Inequality得证
\subsection{二阶强凸条件}
因为$g(x)$为凸函数,故有$\triangledown ^{2} g(x) \geq 0$

代入$g(x) = f(x) - \frac{m}{2}x^{T}x$,有$\triangledown^{2}(f(x) - \frac{m}{2}x^{T}x) = \triangledown^{2} f(x) - mI \geq 0$

即$\triangledown ^{2} f(x) \geq mI$,得证
\section{无穷范数的对偶范数是一范数}
对偶范数定义:$\Vert u\Vert_{*}=\underset{v\neq 0}{sup}\frac{u^{T}v}{\Vert v\Vert}=\underset{\Vert v\Vert=1}{sup}u^{T}v$

再结合霍尔德不等式,若有$\frac{1}{p}+\frac{1}{q}=1$

则$z^{T}x\leq \Vert x \Vert_{p} \Vert z \Vert_{q}$

$\Vert z\Vert_{*}=\underset{x\neq 0}{sup}\frac{z^{T}x}{\Vert x\Vert}_{p}=\Vert z\Vert_{q}$

当$p=1$时,根据霍尔德不等式,$q=\infty $

得证
\section{临近算子}
\subsection{二次函数}
$u = prox^{th}(x)$最优性条件为

$x - u \in t\partial h(x)$

即$x -u \in t(Au + b)$

即$x-u = tAu + tb$

得到$u = (x-tb)(I+tA)^{-1}$
\subsection{负自然对数的和}
有$h(x) = -\sum_{n = 1}^{n} lnx_{i}$,$u_{i} = prox_{th}(x_{i})$最优性条件为

$x- u \in t\partial h(u_{i})$

即$x - u_{i}+\frac{t}{u_{i}} = 0$,解得$u_{i} = \frac{x_{i} + \sqrt{x_{i}^{2} + 4t} }{2}$
\section{临近算子基本运算规则}
\subsection{Linear Combination}
$prox_{f}(x) = \arg\min_{z}\{ \frac{1}{2} \| z- x\| ^{2}_{2} + f(z)\}$

代入$f(x) = ag(x) + b$,有$prox_{f(x)} = \arg\min_{z}\{ \frac{1}{2} \| z- x\| ^{2}_{2} + ag(z) + b\}$

即$f(x) = ag(x) + b$,$prox_{f(x)} = \arg\min_{z}\{ \frac{1}{2} \| z- x\| ^{2}_{2} + ag(z)\}=prox_{ag}(x)$

得证
\subsection{Affine Addition}
$prox_{f}(x) = \arg\min_{z}\{ \frac{1}{2} \| z - x\| ^{2}_{2} + a^{T}z +g(z) \}$

$= \arg\min_{z}\{ \frac{1}{2} \| z\|^{2}_{2} - <z,x> + a^{T}z + g(z) \}$

$ = \arg\min_{z}\{ \frac{1}{2} \| z\|^{2}_{2} - <z,x-a> + g(z) \}$

$= \arg\min_{z}\{ \frac{1}{2} \| z - (x-a)\|^{2}_{2} + g(z) \}$

$= prox_{g}(x-a)$

得证
\subsection{Quadratic Addition}
$prox_(f)(x) = \arg\min_{z}\{ \frac{1}{2} \| z - x\| ^{2}_{2} +g(z)+\frac{\rho }{2} \| z - a\|_{2}^{2}  \}$

$= \arg\min_{z}\{ \frac{1 + \rho}{2} \| z - x\| ^{2}_{2}- <z,x+\rho a > + g(z)\}$

$= \arg\min_{z}\{ \frac{1 + \rho}{2} \| z - x\| ^{2}_{2} - \frac{1}{1 + \rho}<z,x+\rho  a> + \frac{1}{1+\rho}g(z) \}$

$ = \arg\min_{z}\{ \frac{1 + \rho}{2} \| z - (\frac{1}{1+\rho}x + \frac{\rho}{1 + \rho}a)\| ^{2}_{2}+\frac{1}{1 + \rho}g(z)\}$

$ = prox_{\frac{1}{1+\rho}g}(\frac{1}{1+\rho}x + \frac{\rho}{1 + \rho}a)$

得证

\subsection{Scaling and Translation}
如果$f(\mathbf  x )=g (a\mathbf  x +\mathbf b)+$有$ a\neq 0$ 那么

$prox_{f}\left ( \mathbf x \right )=arg\min_{z}\left \{ \frac{1}{2}\Vert \mathbf z - \mathbf x \Vert _2^{2}+f \left (\mathbf z \right )\right \}$

$=arg\min_{z}\left \{ \frac{1}{2}\Vert \mathbf z - \mathbf x \Vert _2^{2}+g\left (a\mathbf z+\mathbf b \right )\right \}$

$=arg\min_{z}\left \{ \frac{1}{2}\Vert \mathbf z \Vert _2^{2}-\left \langle \mathbf z ,\mathbf x \right \rangle+g\left (a\mathbf z+\mathbf b \right )\right \}$

若$\mathbf t=a\mathbf z+b$那么

$\mathbf t=arg\min_{z}\left \{ \frac{1}{2}\Vert \frac{\mathbf t-\mathbf b}{a} - \mathbf x \Vert _2^{2}+g \left (\mathbf t \right )\right \}$

$=arg\min_{z}\left \{ \frac{1}{2a^{2}}\Vert \mathbf t - \left ( a\mathbf x +\mathbf b \right )\Vert _2^{2}+g \left (\mathbf t \right )\right \}$

$=arg\min_{z}\left \{ \frac{1}{2}\Vert \mathbf t - \left ( a\mathbf x +\mathbf b \right )\Vert _2^{2}+a^{2}g \left (\mathbf t \right )\right \}$

$=prox_{a^{2}g}\left ( a\mathbf x+\mathbf b\right )$

$prox_{f}\left ( \mathbf x \right )=\mathbf z=\frac{\mathbf t-\mathbf b}{a}=\frac{1}{a}\left [prox_{a^{2}g}\left ( a\mathbf x+\mathbf b\right )-\mathbf b  \right ]$

得证
\section{优化方法总结}

% Please add the following required packages to your document preamble:
% \usepackage[normalem]{ulem}
% \useunder{\uline}{\ul}{}
\begin{table}[h]
    \centering
    \begin{tabular}{llll}
    \hline
    {\ul } & 梯度法                                       & 次梯度法                                          & 近似点梯度法                                    \\ \hline
    步数     & $O\left ( \frac{1}{k} \right )$           & $O\left ( {k}^{\frac{1}{2}} \right )$         & $O\left ( \frac{1}{k} \right )$           \\
    收敛速率   & $O\left ( \frac{1}{\varepsilon} \right )$ & $O\left ( \frac{1}{\varepsilon^{2}} \right )$ & $O\left ( \frac{1}{\varepsilon} \right )$ \\
    若强凸    & $O\left ( log\left ( \frac{1}{\varepsilon  } \right ) \right )$                          &$O\left ( \frac{1}{\varepsilon} \right )$                        &$O\left ( log\left ( \frac{1}{\varepsilon  } \right ) \right )$                                           \\ \hline
    \end{tabular}
    \end{table}
\section{共轭函数总是凸函数}

可以看成是一系列关于 y 的凸函数取上确界

等价于要证明:$f^{*}\left ( \theta t_{1}+\left ( 1-\theta  \right )t_{2} \right )\leq \theta f^{*}\left ( t_{1} \right )+\left ( 1-\theta  \right )f^{*}\left ( t_{2} \right )$

代入共轭函数的定义,上述不等式等价于:

$\underset{x \in  dom\left ( f \right )}{max}\left \{\left ( \theta t_{1}+\left ( 1-\theta  \right )t_{2} \right )x-f\left ( x \right )  \right \}\leq \theta \underset{x \in  dom\left ( f \right )}{max}\left \{t_{1} x-f\left ( x \right )  \right \} +\left ( 1-\theta  \right ) \underset{x \in  dom\left ( f \right )}{max}\left \{t_{2} x-f\left ( x \right )  \right \}$

左式等价于:

$\underset{x \in  dom\left ( f \right )}{max}\left \{ \theta \left ( t_{1}x-f\left ( x \right ) \right )+\left ( 1-\theta  \right ) \left ( t_{2}x-f\left ( x \right ) \right ) \right \}$

假设在$x_{0}$取到最大值,即:$\theta \left ( t_{1}x_{0}-f\left ( x_{0} \right ) \right )+\left ( 1-\theta  \right ) \left ( t_{2}x_{0}-f\left ( x_{0} \right ) \right ) $

又:

$\theta \left ( t_{1}x_{0}-f\left ( x_{0} \right ) \right )\leq  \underset{x \in  dom\left ( f \right )}{max}\left \{ \theta \left ( t_{1}x_{0}-f\left ( x_{0} \right ) \right ) \right \} $

$\left ( 1-\theta  \right ) \left ( t_{2}x_{0}-f\left ( x_{0} \right ) \right )\leq  \underset{x \in  dom\left ( f \right )}{max}\left \{ \left ( 1-\theta  \right )  \left ( t_{2}x_{0}-f\left ( x_{0} \right ) \right ) \right \} $

所以原式得证

\section{Lagrange对偶问题总是凸优化问题}
原问题:

$\underset{x}{min}f\left ( x \right )$

$s.t.\; g_{i}\left ( x \right )\leq 0,i=1,2,\cdots \cdots ,m$

$h_{j}\left ( x \right )=0,j=1,2,\cdots \cdots ,p$

进行一次拉格朗日乘子法,转换为:

$\underset{x,\lambda ,\nu }{min}L\left ( x,\lambda ,\nu  \right )=f\left ( x \right )+\sum_{i=1}^{m}\lambda _{i}g_{i}\left ( x \right )+\sum_{i=1}^{p}\nu  _{i}h_{i}\left ( x \right )$

再求一次对偶问题:

$\underset{\lambda ,\nu }{max}\underset{x }{min}L\left ( x,\lambda ,\nu  \right )$

$s.t.\; \lambda \geq 0$

首先约束条件一定是凸函数,因为它是线性的,也就是说其实我们现在只要证明:$\underset{x }{min}L\left ( x,\lambda ,\nu  \right )$是凸函数即可

$-\underset{x }{min}L\left ( x,\lambda ,\nu  \right )=-k_{1}\lambda -k_{2}\nu -b$ if $k_{1}=g\left ( x \right ),k_{2}=h\left ( x \right ),b=f\left ( x \right )$

上面跟x xx有关的全部是一个定值。这就是一个线性变化,或者说仿射变换。然后就用到了我们前面用到的那个重要定理:仿射集一定是凸集。所以说这是一个凸函数。

\section{等式约束范数极小化}
$f_{0}^{*}\left ( y \right )=\underset{x}{sup}\left ( y^{T} x -\Vert x \Vert\right )$

根据:$\Vert v \Vert_{*}=sup_{\Vert u \Vert\leq 1}u^{T}v is dual norm of \Vert \cdot  \Vert$

又:$\Vert x \Vert=\underset{x}{sup}\left ( y^{T} x \right ),\Vert y  \Vert_{*}\leq 1$

如果$\Vert y  \Vert_{*}\leq 1,  y^{T} x \leq \Vert x  \Vert ,$ 显然x=0时取等号

又:$\Vert y \Vert_{*}=\underset{x}{sup}\left ( x^{T} y \right )> 1,\Vert x  \Vert\leq 1$

如果$\Vert y  \Vert_{*}> 1,  \exists x , \Vert x \Vert \leq 1,then x^{T}y>1$

$f_{0}^{*}\left ( y \right )\geq y^{T}\left ( tx \right )-\Vert tx  \Vert=t\left ( y^{T}x-\Vert x  \Vert \right )> 0$

$t\rightarrow \infty, f_{0}^{*}\left ( y \right )\rightarrow \infty$

所以:$f_{0}^{*}\left ( y \right )=\left\{\begin{matrix}
    0\; \;\Vert y \Vert_{*}\leq 1 \\ 
    \infty\; \; otherwise
    \end{matrix}\right.$
等式约束范数极小化得证
\section{SVM的KKT条件}

原问题


\begin{equation}
\begin{aligned}
    \min\underset{\beta,\beta_{0},\xi}{} &\frac{1}{2} \| \beta\|_{2}^{2} + C\sum_{i = 1}^{n} \xi_{i} \\
    s.t. &\xi_{i} \geq 0 i = 1,2,3...n\\
         &y_{i}(x_{i}^{T}\beta +\beta_{0}) \geq 1 - \xi_{i}
\end{aligned}
\end{equation} 

稳定性条件:$0\in \partial_{\beta,\xi}(\frac{1}{2} \|\beta\|_{2}^{2} + C\sum_{i = 1}^{n} \xi_{i} + u_{i}(1-\xi_{i} -y_{i}(x_{i} ^{T}\beta +\beta_{0})+\xi_{i}))$

互补松弛条件:$u_{i}(1-\xi_{i} -y_{i}(x_{i} ^{T}\beta +\beta_{0})) = 0$

原可行性:$1-\xi_{i} -y_{i}(x_{i} ^{T}\beta +\beta_{0}) \leq 0$

对偶可行性:$u_i \geq 0$

\section{Rigde Regression的对偶问题}

原岭回归问题,引入约束条件

\begin{equation}
\begin{aligned}
        \min\underset{x}{} &\frac{1}{2} \| Ax - b\|_{2}^{2} + \| x\|_2^2 \\
        s.t. &x = z\\
\end{aligned}
\end{equation}
有$g(u) = \min_x \frac{1}{2} \|Ax -b \|_2^2 + \| z\|_2^2 + u^T(z-x)$

$=\min_x \frac{1}{2} \| Ax - b\|_2^2 - u^{T}x + \|z\|_2^2 + u^{T}z $

$=\min_x \frac{1}{2} \| Ax - b\|_2^2 - ((A^{-1})^{T}u)^T(Ax - b) +\| z\|_2^2 + u^{T}z$

$= - \| (A^{-1})^Tu\|_{2*}^2 - \| -u\|_{2*}^2 - u^TA^{-1}b$

若有$\| (A^{-1})^Tu \| _* \leq 1, \| -u\|_* \leq 1$

则$g(u) = -u^TA^{-1}b$

对偶问题则为$\max_{u} -u^TA^{-1}b$

\section{逻辑回归loss为凸函数}
逻辑岭回归

$\min_{\beta_0, \beta} \sum_{i = 1}^{n}[-y_i(\beta_0 + \beta^{T}x_i) + log(1+e^{\beta_0 + \beta^Tx_i})] + \lambda \|\beta \|_2^2$

其中$y_i \in \{ 0,1\}, x_i\in R^p$

记$R = -y_i(\beta_0 + \beta^{T}x_i) + log(1+e^{\beta_0 + \beta^Tx_i})$

$\frac{\partial R}{\partial \beta_0} = -\sum_{i = 1}^{n} y_i + \sum_{i = 1}^{n} \frac{e^{\beta_0 + \beta^Tx_i}}{1+e^{\beta_0 + \beta^Tx_i}}$

$\frac {\partial^2 R }{\partial \beta_0^2} = \sum_{i = 1}^{n} \frac {e^{\beta_0 + \beta^Tx_i}}{(1+e^{\beta_0 + \beta^Tx_i})^2}$

$\frac{\partial R}{\partial \beta} = - \sum_{i = 1}{n}x_iy_i + \sum_{i = 1}^{n} \frac{x_ie^{\beta_0 + \beta^Tx_i}}{1+e^{\beta_0 + \beta^Tx_i}}$\

$\frac{\partial^2 R}{\partial \beta^2} = \sum_{i = 1}^{n} \frac{x_ie^{\beta_0 + \beta^Tx_i}}{(1+e^{\beta_0 + \beta^Tx_i})^2}$

$\frac{\partial^2 R}{\partial \beta_0 \partial \beta} = \frac{\partial^2 R}{\partial \beta_0 \partial \beta_0} = \sum_{i = 1}^{n} \frac {x_ie^{\beta_0 + \beta^Tx_i}}{(1+e^{\beta_0 + \beta^Tx_i})^2}$

设$t_i = \frac{e^{\beta_0 + \beta^Tx_i}}{(1+e^{\beta_0 + \beta^Tx_i})^2} \geq 0$,则逻辑岭回归Hessian矩阵为

$
H=
\begin{pmatrix}
  \sum_{i = 1}^{n} t_i  ,& \sum_{i=1}^{n} x_it_i\\ 
  \sum_{i =1}^{n} x_it_i ,& \sum_{i=1}^{n}x_i^2t_i
\end{pmatrix}
$

$y^THy = \sum_{i=1}^{n} (x_i +1)^2t_iy^Tt \geq 0$

即Hessian矩阵半正定,故凸函数得证。

\section{Self-Concordant函数性质与实例}
\subsection{自和谐函数性质}
已知为$f(x)$自和谐函数,则做仿射变换$x = ay+b$,令$f(y) = f(ay+b)$

$f^{''}(y) = \alpha^2f^{''}(x), f^{'''}(y) = \alpha^3f^{'''}(x)$

$f(x)$为自和谐函数,有

$|f^{'''}(x)| \leq 2(f^{''}(x)^{\frac{3}{2}})$

两边同时乘$\alpha^3 (\alpha > 0)$,得

$|\alpha^3 f^{'''}(x)| \leq 2(\alpha^2f^{''}(x)^{\frac{3}{2}})$

$|f^{'''}(x)| \leq |2(f^{''}(y))^{\frac {3}{2}}|$

自和谐函数仿射变换之后仍为自和谐函数,affine function成立
\subsection{自和谐函数实例}
\subsubsection{第一个}

内生函数是关于$x$的仿射变换,为自和谐函数,外层函数也为自和谐函数,故整个函数为自和谐函数。

\subsubsection{第二个}

令$g(t) = f(X + tV)$经过分解

$g(t) = -logdet(X + tV)$

$=-logdetX -logdet(I+tX^{-\frac{1}{2}}VX^{-\frac{1}{2}})$

$=-logdetX - \sum_{i =1}^{n} log(1 + t\lambda_i)$,$\lambda_i$为$X^{-\frac{1}{2}}VX^{-\frac{1}{2}}$的特征值

$g(t)$内外层函数均为自和谐函数,故$g(t)$为自和谐函数,故原函数为自和谐函数。
\subsubsection{第三个}
$f(x) = -log(y^2 - x^Tx)$

$= -log(y^2-x^Tx)-logx-log\frac{1}{x}$

$= -log(\frac{y^2}{x}) - logx$

令$g(x) = \frac{y^2}{x} - x$

$g^{'}(x) = - 1-\frac{y^2}{x^2}$

$g^{''}(x) = \frac{2y^T}{x^3}$

$g^{'''}(x) = -\frac{6y^2}{x^4}$

有$|g^{'''}(x)| \leq 3\frac{g^{''}(x)}{x}$

由此$f(x)$为自和谐函数
\end{document}